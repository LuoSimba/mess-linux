\documentclass{mynote}

\begin{document}

\section{set}
\begin{BashCode}
type set
# set is a shell builtin
\end{BashCode}

\texttt{set} 用于设置 Bash 的选项(shell options)和位置参数(positional parameters)。

\subsection{设置位置参数}

\begin{BashCode}
set -- # 清空参数
echo $# # 结果为 0 因为没有参数

set -- aaa
echo $# # 结果为 1

set -- aaa bbb
echo $# # 结果为 2

set -- aaa bbb ccc
echo "$@"
# 输出:aaa bbb ccc
for i; do
	echo "$i"
done
# 输出三行:
# aaa
# bbb
# ccc
\end{BashCode}

% ----------------------------------
\section{for}

\begin{BashCode}
type for
# for is a shell keyword
\end{BashCode}

\texttt{for} 是一个 Bash 的关键词。我们可以使用 

\begin{BashCode}
help for
man bash
\end{BashCode}

\noindent 来查看 \texttt{for} 的用法。

\texttt{for} 可以省略 \texttt{in words} 部分,变成:

\begin{BashCode}
for i do echo "$i"; done

for i; do echo "$i"; done
\end{BashCode}

这种形式等价于:

\begin{BashCode}
# if `in WORDS ...;' is not present, then `in "$@"' is assumed.
for i in "$@"; do echo "$i"; done
\end{BashCode}

\texttt{do} 之前的分号是可选的,以上两种形式都是对的。但是 \texttt{done} 之前的分号是必须的。或者 \texttt{done} 单独写一行:

\begin{BashCode}
for i do echo "$i"
done
\end{BashCode}

\texttt{help} 和 \texttt{man} 给出的语法(细节)都是不对的,是简化版的。以 \texttt{gnu.org} 官网文档为准。


% -----------------------
\section{关于 \$@}

从表面上看,\BashCodeInline{"$@"} 是一个由双引号包裹的字符串。\BashCodeInline{$@} 是字符串内部放置的一个变量。但实际上它不是\textbf{一个}字符串。

为了便于观察,我们需要先定义一个函数:

\begin{BashCode}
foo () {
	echo "size: $#"      # 先输出参数个数
	for i; do
		echo "-$i-"     # 再输出每一个参数
	done
}
\end{BashCode}

我们先来看一下空参数的情况:

\begin{BashCode}
set -- # 清空参数

# 我们会觉得 "$@" 是空字符串,试一下:
test "$@" = ""  # 报错:unary operator expected
test = ""       # 报错:(与上一个命令同样的错误)
# 从这里我们可以看出,它与空字符串在语法上是不等价的
# 它不是空字符串,更像是消失了。

# 我们再用函数观察:
foo ""
# 输出如下:
# size: 1
# --

# 以上,说明空字符串也算作一个参数

foo "$@"
# 输出如下:
# size: 0

# 以上,说明它不会传入任何参数(消失了)。
\end{BashCode}

我们再看一下只有一个参数的情况:

\begin{BashCode}
set -- aaa # 设置一个参数

test "$@" = "aaa" # 结果是真

foo "aaa"
foo "$@"
# 两个命令输出完全一样:
# size: 1
# -aaa-
\end{BashCode}

然后再看看两个参数的情况:

\begin{BashCode}
set -- aaa bbb

echo "$@"
# 输出结果看起来是 "aaa bbb"

# 我们把输出结果存入文件试试:
echo "$@"      > a.txt
echo "aaa bbb" > b.txt
diff a.txt b.txt
# 比较的结果是:两个文件内容一模一样

# 那么是不是说明它就是字符串 "aaa bbb" 呢?
test        "$@" = "aaa bbb"  # 报错:too many arguments
test "aaa" "bbb" = "aaa bbb"  # 报错:too many arguments
# 它们不但不相等,而且还引起了语法错误
# 我们猜测它不是 "aaa bbb" 而是分裂成了两个字符串 "aaa" "bbb"

echo "aaa" "bbb" > a.txt
echo "aaa bbb"   > b.txt
diff a.txt b.txt
# 这样比较的结果也是:两个文件内容一模一样

# 我们再用函数观察一下:
foo "aaa bbb"
# 输出:
# size: 1
# -aaa bbb-
foo "aaa" "bbb"
# 输出:
# size: 2
# -aaa-
# -bbb-
foo "$@"
# 输出:
# size: 2
# -aaa-
# -bbb-

# 所以 "$@" 其实等价于 "aaa" "bbb" 两个参数的效果
\end{BashCode}

从视觉上观察,它是一个字符串。但实际运行的结果表明,它不是一个字符串,而是不定个数的字符串。

我们这里只尝试了 \BashCodeInline{"$@"} 本身的情况,如果字符串里含有其他成分,比如像 \BashCodeInline{"hello $@"} 这样,结果又会有些不同。

\section{不要在循环中用命令}

以下代码是在当前文件夹创建大量的文件,不过它的运行速度特别慢:

\begin{BashCode}
# 耗时 7 分 52 秒
time for i in {1..250000}; do
	touch "$i.txt"
done
\end{BashCode}

一开始我以为是 Bash 的循环速度很慢,所以我先测试一下空循环的速度:

\begin{BashCode}
# 耗时 0.5 秒
time for i in {1..250000}; do
	:
done
\end{BashCode}

发现执行单纯的循环,Bash 的速度还可以。然后想是不是 \texttt{touch} 命令的效率问题,所以用 \texttt{rm} 命令再试试效果:

\begin{BashCode}
# 耗时 8 分 06 秒
time for i in {1..250000}; do
	rm "$i.txt"
done
\end{BashCode}

\texttt{rm} 也很慢,\texttt{touch}也很慢,所以不会是 \texttt{touch} 一个命令的问题。到这里,怀疑的是所有的文件操作(创建、删除)命令都会很慢。

然后用 Python 来试试创建文件:

\begin{PythonCode}
import os

# 耗时 3.6 秒
for i in range(250000):
	os.mknod(str(i) + ".txt")
\end{PythonCode}

Python 非常快,只需要 3.6 秒就创建了 25 万个空文件。

所以,问题也不是文件操作引起的。

至此,怀疑是在循环中频繁的创建进程、销毁进程,才导致的速度缓慢。因为一个 \texttt{touch} 命令、或者一个 \texttt{rm} 命令,Bash 都要创建一个子进程来执行。这样一来,最终要创建 25 万个进程。这么多的进程最终导致任务缓慢。我们来验证一下。先创建一个空白的可执行文件:

\begin{CCode}
// 文件: a.c
// 编译:gcc -o a.out a.c
/**
 * 什么也不做
 */
int main()
{
	return 0;
}
\end{CCode}

然后我们在 Bash 循环中执行这个可执行文件:

\begin{BashCode}
# 耗时 6 分 55 秒
time for i in {1..250000}; do
	./a.out
done
\end{BashCode}

这个循环执行将近 7 分钟。所以确实是 \textbf{创建进程} 消耗了很多时间。而 \textbf{创建文件、删除文件} 等操作的耗时反而可以忽略不记。

所以,应该避免在一个很大的循环中频繁地调用外部命令。




\end{document}