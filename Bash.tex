\documentclass{mynote}

\begin{document}

\section{set}
\begin{BashCode}
type set
# set is a shell builtin
\end{BashCode}

\texttt{set} 用于设置 Bash 的选项(shell options)和位置参数(positional parameters)。

\subsection{设置位置参数}

\begin{BashCode}
set -- # 清空参数
echo $# # 结果为 0 因为没有参数

set -- aaa
echo $# # 结果为 1

set -- aaa bbb
echo $# # 结果为 2

set -- aaa bbb ccc
echo "$@"
# 输出:aaa bbb ccc
for i; do
	echo "$i"
done
# 输出三行:
# aaa
# bbb
# ccc
\end{BashCode}

% ----------------------------------
\section{for}

\begin{BashCode}
type for
# for is a shell keyword
\end{BashCode}

\texttt{for} 是一个 Bash 的关键词。我们可以使用 

\begin{BashCode}
help for
man bash
\end{BashCode}

\noindent 来查看 \texttt{for} 的用法。

\texttt{for} 可以省略 \texttt{in words} 部分,变成:

\begin{BashCode}
for i do echo "$i"; done

for i; do echo "$i"; done
\end{BashCode}

这种形式等价于:

\begin{BashCode}
# if `in WORDS ...;' is not present, then `in "$@"' is assumed.
for i in "$@"; do echo "$i"; done
\end{BashCode}

\texttt{do} 之前的分号是可选的,以上两种形式都是对的。但是 \texttt{done} 之前的分号是必须的。或者 \texttt{done} 单独写一行:

\begin{BashCode}
for i do echo "$i"
done
\end{BashCode}

\texttt{help} 和 \texttt{man} 给出的语法(细节)都是不对的,是简化版的。以 \texttt{gnu.org} 官网文档为准。

\end{document}